\chapter{\IfLanguageName{dutch}{literatuurstudie}{Introduction}}%
\label{ch:literatuurstudie}


In een wereld vol onverwachte verstoringen is business continuity essentieel voor bedrijven om hun activiteiten, reputatie en klantvertrouwen te beschermen en concurrerend te blijven. Business continuity (BC) verwijst naar een verzameling methoden en procedures waarmee een organisatie kan zorgen dat haar activiteiten blijven doorgaan, zelfs als er een storing optreedt. Storingen kunnen van uiteenlopende aard zijn – van kleine incidenten zoals een kapotte netwerkkabel tot grootschalige rampen zoals overstromingen of aardbevingen. Omdat het type storing enorm kan variëren, is er geen uniforme oplossing die voor elk scenario werkt. In plaats daarvan maken organisaties gebruik van een combinatie van strategieën en technologieën om hun bedrijfscontinuïteit te waarborgen. \autocite{Zhu2015}

Een belangrijk onderdeel van business continuity is het concept van hoge beschikbaarheid. Dit betekent dat de bedrijfssystemen zo ontworpen zijn dat de werkzaamheden kunnen doorgaan, zelfs als er een onderdeel faalt. Denk bijvoorbeeld aan redundante systemen, waarbij kritieke apparatuur of netwerken gedupliceerd zijn, zodat er altijd een back-up is als een onderdeel uitvalt.

Disaster recovery (DR) is een specifiek onderdeel binnen business continuity dat zich richt op het herstelproces na een grote ramp. Waar business continuity breder kijkt naar preventieve maatregelen om continuïteit te waarborgen, richt disaster recovery zich op de acties en hulpmiddelen die nodig zijn om na een grote verstoring of ramp zo snel mogelijk weer operationeel te zijn. Het doel van disaster recovery is om de schade van een verstoring te beperken en het bedrijf zo snel mogelijk terug in zijn normale toestand te brengen.