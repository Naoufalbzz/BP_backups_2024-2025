%==============================================================================
% Sjabloon onderzoeksvoorstel bachproef
%==============================================================================
% Gebaseerd op document class `hogent-article'
% zie <https://github.com/HoGentTIN/latex-hogent-article>

% Voor een voorstel in het Engels: voeg de documentclass-optie [english] toe.
% Let op: kan enkel na toestemming van de bachelorproefcoördinator!
\documentclass{hogent-article}
\usepackage{graphicx}
\usepackage[backend=biber]{biblatex}
% Invoegen bibliografiebestand
\addbibresource{voorstel.bib}

% Informatie over de opleiding, het vak en soort opdracht
\studyprogramme{Professionele bachelor toegepaste informatica}
\course{Bachelorproef}
\assignmenttype{Onderzoeksvoorstel}
% Voor een voorstel in het Engels, haal de volgende 3 regels uit commentaar
% \studyprogramme{Bachelor of applied information technology}
% \course{Bachelor thesis}
% \assignmenttype{Research proposal}

\academicyear{2024-2025} % TODO: pas het academiejaar aan

% TODO: Werktitel
\title{Optimalisatie van de back-upstrategie voor Azure PostgreSQL en MySQL databases bij Forvis Mazars: Een proof of concept met immutabele opslag en automatische back-ups}

% TODO: Studentnaam en emailadres invullen
\author{Naoufal Bouazzaoui}
\email{naoufal.bouazzaoui@student.hogent.be}

% TODO: Geef de co-promotor op
\supervisor[Co-promotor]{R. Tetaert (Forvis Mazars, \href{mailto:remy.tetaert@mazars.be}{remy.tetaert@mazars.be})}

% Binnen welke specialisatierichting uit 3TI situeert dit onderzoek zich?
% Kies uit deze lijst:
%
% - Mobile \& Enterprise development
% - AI \& Data Engineering
% - Functional \& Business Analysis
% - System \& Network Administrator
% - Mainframe Expert
% - Als het onderzoek niet past binnen een van deze domeinen specifieer je deze
%   zelf
%
\specialisation{System \& Network Administrator}
\keywords{Back-ups, Security, Ransomware}

\begin{document}

\begin{abstract}
In deze bachelorproef wordt een optimalisatie van de back-upstrategie voor de Azure PostgreSQL en MySQL databases bij Forvis Mazars onderzocht, de focus ligt voornamelijk op immutabele opslag en automatische back-ups. Het doel is om de back-upstrategie van Forvis Mazars te optimaliseren en het resistent te maken tegen ransomware-aanvallen. Daarnaast wordt er ook een Proof-of-Concept (PoC) uitgevoerd, waarin immutabele opslag wordt geïmplementeerd om ervoor te zorgen dat back-ups onveranderlijk zijn na opslag. Verder worden geautomatiseerde back-ups geïmplementeerd om de back-ups efficiënter te maken en de consistentie van de back-ups te verbeteren. In de state-of-the-art ligt de focus op bestaande back-upstrategieën, zoals cloud back-ups, on-premise back-ups, en offline back-ups.   De methodologie omvat een literatuurstudie, een analyse van de huidige back-upstrategie bij Forvis Mazars, en de ontwikkeling van een PoC. De verwachte resultaten zullen de verbeterde beveiliging en efficiëntie van de back-upstrategie aantonen, met als doel het minimaliseren van de risico's op dataverlies en het waarborgen van bedrijfscontinuïteit.
\end{abstract}

\tableofcontents

% De hoofdtekst van het voorstel zit in een apart bestand, zodat het makkelijk
% kan opgenomen worden in de bijlagen van de bachelorproef zelf.
\input{voorstel-inhoud}

\printbibliography[heading=bibintoc]

\end{document}