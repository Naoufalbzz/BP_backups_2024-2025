%%=============================================================================
%% Voorwoord
%%=============================================================================

\chapter*{\IfLanguageName{dutch}{Woord vooraf}{Preface}}%
\label{ch:voorwoord}

%% TODO:
%% Het voorwoord is het enige deel van de bachelorproef waar je vanuit je
%% eigen standpunt (``ik-vorm'') mag schrijven. Je kan hier bv. motiveren
%% waarom jij het onderwerp wil bespreken.
%% Vergeet ook niet te bedanken wie je geholpen/gesteund/... heeft

Met trots en voldoening presenteer ik deze bachelorproef, het sluitstuk van mijn opleiding Toegepaste Informatica aan de HoGent. In het begin toen ik begon met zoeken naar een onderwerp had ik geen idee waar ik moest beginnen en ik vond dit dus ook een grote uitdaging. Ik begon met het opzoeken naar bepaalde interessante onderwerpen en ik kwam uiteindelijk op dit onderwerp. Ik heb sterke interesses in cybersecurity dus ik wou dit implementeren in mijn bachelorproef. Daarbij ontwikkelde ik ook interesse in back-ups tijdens het opleidingsonderdeel Cybersecurity Advanced. Het schrijven van deze bachelorproef was niet alleen een uitdagend leerproces, maar ook een unieke kans om mijn interesse in systeembeheer en cybersecurity verder te verdiepen. Dit project heeft me niet alleen geholpen mijn technische kennis te versterken, maar ook om praktijkervaring op te doen binnen een professionele context.

Ik wil graag mijn dankbaarheid uitdrukken aan mijn co-promotor, Rémy Tetaert, voor zijn waardevolle begeleiding, expertise en tijd tijdens dit proces. Zijn inzichten, ondersteuning en begeleiding waren heel belangrijk voor mij. Daarnaast wil ik mijn promotor, Martijn Saelens, bedanken voor zijn constructieve feedback en kritische blik, die me steeds hebben geholpen om mijn werk te verbeteren. 

Tot slot wil ik ook mijn familie en vrienden bedanken voor hun geduld, steun en motivatie gedurende deze periode. Zonder hen zou dit niet mogelijk zijn geweest.

