\chapter{\IfLanguageName{dutch}{literatuurstudie}{Introduction}}%
\label{ch:Literatuurstudie}

\subsection{Back-ups in het kader van bedrijfscontinuïteit en disaster recovery}
Bedrijfscontinuïteit verwijst naar de aanpak en procedures dat een bedrijf gebruikt om de voortgang van zijn werkzaamheden te bewaren, zelfs in het geval van incidenten. Deze incidenten kunnen variëren van relatief kleine problemen, zoals een gebroken netwerkverbinding, tot grote natuurrampen zoals een aardbevingen. Omdat er zoveel soorten incidenten kunnen gebeuren is het moeilijk om een oplossing te vinden die ervoor zorgt dat bedrijven in alle gevallen beschermt zijn. In plaats daarvan gebruiken bedrijven een mix van strategieën en technologieën om de continuïteit van hun processen te beschermen. De 2 belangrijkste concepten voor de bedrijfscontinuïteit zijn hoge beschikbaarheid en disaster recovery.

Hoge beschikbaarheid duidt op het feit dat een bedrijf zodanig is ingericht dat het kan blijven draaien, zelfs als bepaalde systemen of componenten uitvallen. Een voorbeeld hiervan zijn twee routers die zijn geconfigureerd in een actieve-passieve opstelling. In deze configuratie is één router de primaire router die al het inkomende en uitgaande verkeer verwerkt, terwijl de andere router als reserve werkt. In het geval dat de primaire router faalt door een hardwarestoringen of netwerkprobleem, dan neemt de tweede router automatisch de rol van de primaire router over, zonder dat dit merkbare impact heeft op de netwerkverbindingen van de organisatie. Hierdoor blijft de beschikbaarheid van het netwerk gegarandeerd en blijft de downtime laag \autocite{Zhu2015}.

Disaster recovery (DR) is een onderdeel van bedrijfscontinuïteit dat zich specifiek richt op het herstellen van bedrijfsactiviteiten na een incident zoals een cyberaanval of ernstige verstoring. Terwijl bedrijfscontinuïteit zich richt op bredere preventieve maatregelen om de continuïteit te waarborgen, focust disaster recovery zich juist op de praktische stappen en hulpmiddelen die nodig zijn om de organisatie na een verstoring weer snel operationeel te maken. Het doel van disaster recovery is om schade zoveel mogelijk te beperken en de normale gang van zaken zo snel mogelijk te herstellen .
 
Back-ups spelen een belangrijke rol voor de continuïteit van een bedrijf en zijn vaak de eerste stap bij het opstellen van een disaster recovery plan (DRP). Bij een optimale situatie is er na een incident geen data verloren en is alle data relatief snel terug beschikbaar. Indien een bedrijf geen back-ups heeft van de belangrijke data zal de data in het geval van een incident verloren raken. Zonder back-ups zal het ook een grotere uitdaging zijn voor het bedrijf om de normale bedrijfsactiviteiten terug uit te voeren. Een belangrijke doelstelling van een bedrijf is winst maken. In het geval van een incident waarbij de bedrijfsactiviteiten niet normaal kunnen verlopen zal deze doelstelling verhindert worden en zal er dus financieel verlies optreden. Bij specifieke bedreigingen, zoals ransomware-aanvallen spelen ransomware-resistente back-ups een cruciale rol. Door back-ups te beveiligen tegen ransomware-aanvallen kunnen bedrijven hun data herstellen zonder losgeld te betalen. Dit benadrukt het belang van back-ups die niet alleen snel toegankelijk zijn, maar ook bestand zijn tegen digitale bedreigingen \autocite{Ghazi2013}.

\subsection{Back-upmethoden en -technieken}
Back-ups zijn een belangrijk onderdeel van datamanagement en databeveiliging binnen organisaties. Back-ups zorgen voor de continuïteit van bedrijfssystemen in het geval van een incident zoals een cyberaanval. Back-ups zijn momentopnamen van gegevens die op een bepaald tijdstip zijn gemaakt, opgeslagen in een wereldwijd gebruikelijk formaat en gedurende een bepaalde periode van bruikbaarheid worden bijgehouden, waarbij elke volgende kopie van de gegevens onafhankelijk van de eerste wordt bewaard\autocite{Nelson2011}. Door een aparte kopie van de gegevens te bewaren, kunnen bedrijven en individuen na een incident hun systemen of bestanden herstellen naar een eerdere, veilige staat. Hierbij kunnen back-ups zowel volledige datasets als selectieve bestandstypen omvatten, afhankelijk van de strategie en de specifieke behoeften van de organisatie. Back-ups zijn een preventieve maatregel en het doel ervan is om dataverlies tegen te gaan. Dataverlies kan optreden door menselijke fouten, cyberaanvallen, en natuur- of bedrijfsrampen. Daarbij speelt beveiliging een belangrijke rol in een tijd waarin ransomware-aanvallen en datalekken frequenter voorkomen. Door back-ups versleuteld op te slaan en te beveiligen tegen ongeautoriseerde toegang, kunnen bedrijven zich beschermen tegen het verliezen van data. 
\subsection{Ransomware-resistente back-upoplossingen}
\subsection{Back-upmethoden en -technieken}
\subsection{Back-upmethoden en -technieken}
\subsection{Proof of Concept}