\chapter{\IfLanguageName{dutch}{literatuurstudie}{Introduction}}%
\label{ch:Literatuurstudie}

\subsection{Back-ups in het kader van bedrijfscontinuïteit en disaster recovery}
Bedrijfscontinuïteit verwijst naar de aanpak en procedures dat een bedrijf gebruikt om de voortgang van zijn werkzaamheden te bewaren, zelfs in het geval van storingen of ernstige verstoringen. Deze verstoringen kunnen variëren van relatief kleine problemen, zoals een gebroken netwerkverbinding, tot grote natuurrampen zoals aardbevingen. Omdat de aard en impact van verstoringen zo uiteenlopen, is er geen enkele oplossing die in alle gevallen voldoet. In plaats daarvan stellen bedrijven een mix van strategieën en technologieën samen om de continuïteit van hun processen te beschermen.

Een essentieel onderdeel van bedrijfscontinuïteit is het concept van hoge beschikbaarheid. Dit betekent dat een bedrijf zodanig is ingericht dat het kan blijven draaien, zelfs als bepaalde systemen of componenten uitvallen. Denk hierbij aan systemen met een dubbele opstelling, waarbij kritische apparatuur of functies worden gerepliceerd. Als er een onderdeel faalt, kan een reserveonderdeel de taak direct overnemen, wat downtime minimaliseert.

Disaster recovery (DR) is een onderdeel van bedrijfscontinuïteit dat zich specifiek richt op het herstellen van bedrijfsactiviteiten na een zware ramp of ernstige verstoring. Terwijl bedrijfscontinuïteit zich richt op bredere preventieve maatregelen om de continuïteit te waarborgen, focust disaster recovery zich juist op de praktische stappen en hulpmiddelen die nodig zijn om de organisatie na een verstoring weer snel operationeel te maken. Het doel van disaster recovery is om schade zoveel mogelijk te beperken en de normale gang van zaken zo spoedig mogelijk te herstellen. \autocite{Zhu2015}
