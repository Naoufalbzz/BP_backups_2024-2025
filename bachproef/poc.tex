\chapter{Proof-of-concept}%
\label{ch:poc}
Voor het praktische deel van deze bachelorproef werden er drie virtuele machines opgezet binnen VirtualBox met behulp van Vagrant om een gecontroleerde testomgeving te creëren. Deze virtuele machines (VM’s) simuleren een scenario waarin een ransomware-aanval gericht wordt op databases die door het bedrijf worden beheerd. Het primaire doel van deze simulatie is aan te tonen dat het gebruik van immutable storage een effectieve maatregel kan zijn om belangrijke data te beschermen tegen ransomware-aanvallen.

Voor het opzetten van de virtuele machines in de Proof-of-Concept (PoC) werd gebruik gemaakt van een Vagrantfile. De Vagrantfile definieert de specificaties en configuraties van de VM’s, zoals geheugen, CPU, netwerkadapters en besturingssysteem. 
\begin{lstlisting}[language=Ruby, caption={Vagrantfile voor drie VM's: Backup Server, Client, en Attacker}]
Vagrant.configure("2") do |config|

# Define the first VM - Primary
config.vm.define "primary" do |primary|
primary.vm.box = "ubuntu/jammy64"

# Network configuration
primary.vm.network "private_network", ip: "192.168.0.10", virtualbox__intnet: "internal_network"

# Hardware resources
primary.vm.provider "virtualbox" do |vb|
vb.memory = "2048" # 2GB RAM
vb.cpus = 1        # 1 CPU
end
end

# Define the second VM - Backup
config.vm.define "backup" do |backup|
backup.vm.box = "ubuntu/jammy64"

# Network configuration
backup.vm.network "private_network", ip: "192.168.0.20", virtualbox__intnet: "internal_network"

# Hardware resources
backup.vm.provider "virtualbox" do |vb|
vb.memory = "2048" # 2GB RAM
vb.cpus = 1        # 1 CPU
end
end

# Define the third VM - Attacker VM
config.vm.define "attacker" do |additional|
additional.vm.box = "ubuntu/jammy64"

# Network configuration
additional.vm.network "private_network", ip: "192.168.0.30", virtualbox__intnet: "internal_network"

# Hardware resources
additional.vm.provider "virtualbox" do |vb|
vb.memory = "1024" # 1GB RAM
vb.cpus = 1        # 1 CPU
end
end

end
\end{lstlisting}

In de onderstaande tabel worden de specificaties van de drie virtuele machines weergegeven die in de Proof of Concept zijn gebruikt. Elke VM heeft een specifieke functie binnen de gesimuleerde omgeving. De tabel bevat details over de hoeveelheid toegewezen RAM, het aantal CPU-cores, het gebruikte besturingssysteem, de toegewezen IP-adressen en de configuratie van de netwerkadapter. Deze configuratie zorgt ervoor dat de VM's binnen hetzelfde interne netwerk met elkaar kunnen communiceren, wat essentieel is voor het testen van de ransomware-aanval en de back-upstrategieën.
\begin{longtable}{|l|c|c|c|l|l|}
    \hline
    \textbf{Functie} & \textbf{RAM} & \textbf{CPU Cores} & \textbf{IP} & \textbf{Besturingssysteem} & \textbf{Netwerkadapter} \\ \hline
    Primary server    & 2 GB         & 1                  & 192.168.0.10 & Ubuntu 22.04.5 LTS & Private Network \\ \hline
    Back-up server           & 1 GB         & 1                  & 192.168.0.20 & Ubuntu 22.04.5 LTS & Private Network \\ \hline
    Attacker VM         & 2 GB         & 1                  & 192.168.0.30 & Ubuntu 22.04.5 LTS     & Private Network \\ \hline
\end{longtable}