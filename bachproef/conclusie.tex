%%=============================================================================
%% Conclusie
%%=============================================================================

\chapter{Conclusie}%
\label{ch:conclusie}

In het kader van deze bachelorproef werd een robuuste back-upstrategie ontwikkeld en getest om de veiligheid van databases bij Forvis Mazars te waarborgen, met een sterke focus op het beschermen tegen ransomware-aanvallen. De proef werd opgedeeld in vier hoofdonderdelen, die elk bijdroegen aan de evaluatie van de effectiviteit en efficiëntie van de voorgestelde oplossingen.

In het eerste deel van het onderzoek werd een gecontroleerde testomgeving gecreëerd door drie virtuele machines (VM’s) op te zetten met behulp van Vagrant en VirtualBox. Deze VMs simuleerden een ransomware-aanval waarbij de back-upbestanden werden versleuteld en de actieve database op de actieve VM werd verwijderd. Het gebruik van immutable storage bleek effectief in het beschermen van de back-up, aangezien de integriteit van de gegevens behouden bleef. Door het vergelijken van de hashes van de originele en herstelde back-ups werd aangetoond dat de gegevens niet waren aangetast, wat de effectiviteit van immutability in een ransomware-aanval bevestigde.

In het tweede deel werd een Azure-oplossing geïmplementeerd in de actieve productieomgeving van Forvis Mazars, waarbij een immutable Azure Storage-container werd opgezet voor het veilig opslaan van back-ups. Dit deel van het onderzoek bevestigde dat immutable opslag effectief kan bijdragen aan het beschermen van back-ups tegen wijzigingen en dataverlies door externe bedreigingen.

Het derde onderdeel betrof de implementatie van een schaalbare en geautomatiseerde back-upoplossing in een Kubernetes-omgeving. Hiervoor werden PostgreSQL- en MySQL-databases lokaal gehost op een Vagrant VM voor testdoeleinden, terwijl de productieomgeving van Forvis Mazars gebruik maakt van Azure. Door containers binnen Kubernetes Pods te draaien, konden back-ups automatisch worden uitgevoerd met behulp van CronJobs. Daarnaast werden de back-ups opgeslagen in Persistent Volumes (PV) en beveiligd met Kubernetes Secrets. Dit zorgde voor een geautomatiseerde en veilige oplossing die zich goed leent voor schaalbaarheid en integratie met de bestaande infrastructuur van Forvis Mazars.

Het laatste onderdeel van het onderzoek richtte zich op het testen van het herstelproces van back-ups. Zowel de MySQL- als PostgreSQL-databases werden hersteld uit de back-ups, waarbij de integriteit van de gegevens werd gecontroleerd door het vergelijken van de hashes van de data voor en na het herstel. Het herstelproces werd uitgevoerd over een internetsnelheid van 75,33 Mbit/s, waarbij de MySQL-database (383 KiB) binnen 48 seconden werd hersteld, evenals de PostgreSQL-database (45,16 KiB). Het herstel toonde aan dat zowel de snelheid als de integriteit van de gegevens behouden bleef, wat de betrouwbaarheid van het back-up- en herstelproces benadrukte.

Samenvattend heeft deze bachelorproef aangetoond dat het gebruik van immutabele opslag, geautomatiseerde back-ups via Kubernetes en betrouwbare herstelprocessen de veiligheid en effectiviteit van de back-upstrategie bij Forvis Mazars aanzienlijk verbeteren. De voorgestelde oplossing biedt niet alleen bescherming tegen ransomware-aanvallen, maar zorgt ook voor een efficiënte en schaalbare back-upworkflow die snel herstel mogelijk maakt, met behoud van de integriteit van de gegevens.

