%%=============================================================================
%% Inleiding
%%=============================================================================

\chapter{\IfLanguageName{dutch}{Inleiding}{Introduction}}%
\label{ch:inleiding}

De beveiliging van gegevens is van cruciaal belang voor organisaties, vooral gezien de toenemende dreigingen van cyberaanvallen zoals ransomware. Gezien de digitale transformatie die veel bedrijven doormaken, zijn betrouwbare en veilige back-up oplossingen essentieel om de continuïteit van de bedrijfsvoering te waarborgen. Dit geldt in het bijzonder voor bedrijven die werken met cloudplatformen zoals Microsoft Azure, waar databases zoals PostgreSQL en MySQL vaak cruciaal zijn voor het dagelijks functioneren. Bij Forvis Mazars worden momenteel back-ups van Azure-databases gemaakt via een combinatie van automatische volledige back-ups en handmatige back-ups via scripts. Deze aanpak kent echter enkele beperkingen, zoals de onregelmatige uitvoering van de handmatige back-ups en een gebrek aan geautomatiseerde processen, wat de veiligheid en efficiëntie van het systeem in gevaar kan brengen.



\section{\IfLanguageName{dutch}{Probleemstelling}{Problem Statement}}%
\label{sec:probleemstelling}
In deze bachelorproef wordt de huidige back-upstrategie van Forvis Mazars geanalyseerd en geoptimaliseerd. De focus ligt hierbij op het verbeteren van de back-upstrategie voor de Azure PostgreSQL en MySQL databases, met bijzondere aandacht voor ransomware-resistentie en de integratie van immutabele opslagtechnieken. Het doel is om de bestaande strategie te versterken en de kans op dataverlies door cyberaanvallen te minimaliseren. Daarnaast zal er een proof-of-concept worden uitgevoerd om de effectiviteit van immutabele opslag te testen in een scenario waarbij een ransomware-aanval wordt nagebootst op mock-up bestanden. De probleemstelling van dit onderzoek is dat de huidige back-upstrategie bij Forvis Mazars niet voldoende robuust is om het risico op dataverlies door ransomware effectief te mitigeren. Dit onderzoek heeft tot doel de back-upstrategieën van Forvis Mazars te verbeteren door middel van geautomatiseerde processen en door gebruik te maken van immutabele opslag voor extra beveiliging tegen dataverlies. De doelgroep van dit onderzoek bestaat uit Forvis Mazars

\section{\IfLanguageName{dutch}{Onderzoeksvraag}{Research question}}%
\label{sec:onderzoeksvraag}

De onderzoeksvraag van deze bachelorproef luidt:

\textbf{Hoe kan de back-upstrategie voor Azure PostgreSQL en MySQL databases bij Forvis Mazars worden geoptimaliseerd met behulp van immutabele opslag en automatische back-ups?}

\section{Deelvragen}
De onderzoeksvraag kan verder opgedeeld worden in de volgende deelvragen.:
\begin{itemize}
    \item Hoe veilig en betrouwbaar zijn de huidige back-upoplossingen van Forvis Mazars voor Azure PostgreSQL en MySQL databases?
    \item Welke rol speelt immutabele opslag in het beschermen van back-ups tegen ransomware en andere vormen van dataverlies?
    \item Wat zijn de belangrijkste uitdagingen bij het integreren van immutabele opslag met Azure cloud back-upsystemen?
\end{itemize}

\section{\IfLanguageName{dutch}{Onderzoeksdoelstelling}{Research objective}}%
\label{sec:onderzoeksdoelstelling}

Het doel van dit onderzoek is om de back-upstrategie van Forvis Mazars te optimaliseren door de huidige back-upmethoden te analyseren en te verbeteren. Het onderzoek richt zich specifiek op het implementeren van immutabele opslag om de bescherming tegen ransomware-aanvallen te versterken. Daarnaast wordt de automatisering van de manuele back-ups onderzocht en geïmplementeerd, aangezien deze momenteel niet frequent genoeg worden uitgevoerd. Een proof-of-concept (PoC) zal worden uitgevoerd door virtuele machines te gebruiken en een ransomware-aanval na te bootsen om de effectiviteit van de immutabele opslag te testen. Dit proefproject zal verder bijdragen aan het verbeteren van de bestaande automatische back-upstructuur door het toevoegen van meer geautomatiseerde processen, wat de efficiëntie en de veiligheid van het back-upbeheer binnen het bedrijf zal vergroten.
\section{\IfLanguageName{dutch}{Opzet van deze bachelorproef}{Structure of this bachelor thesis}}%
\label{sec:opzet-bachelorproef}

% Het is gebruikelijk aan het einde van de inleiding een overzicht te
% geven van de opbouw van de rest van de tekst. Deze sectie bevat al een aanzet
% die je kan aanvullen/aanpassen in functie van je eigen tekst.

De rest van deze bachelorproef is als volgt opgebouwd:

Hoofdstuk~\ref{ch:stand-van-zaken} biedt een overzicht van de huidige kennis en technologieën rondom back-upstrategieën, ransomware-beveiliging en immutabele opslag. De literatuur helpt de basis te leggen voor het verbeteren van de back-upbeveiliging bij Forvis Mazars.

In hoofdstuk~\ref{ch:methodologie} worden de stappen van het onderzoek beschreven. Een requirementsanalyse werd uitgevoerd om de huidige back-upstrategie van Forvis Mazars te evalueren en verbeterpunten te identificeren. Vervolgens werd de opzet voor een Proof-of-Concept (PoC) uitgewerkt.

Hoofdstuk~\ref{ch:analyse} onderzoekt de huidige back-upstrategie bij Forvis Mazars en stelt verbeteringen voor, zoals de automatisering van handmatige back-ups en het implementeren van immutabele opslag voor verhoogde veiligheid.

In dit hoofdstuk~\ref{ch:poc} wordt de uitvoering van de proof-of-concept beschreven, waarin immutabele opslag wordt getest door een ransomware-aanval na te bootsen op een virtuele machine.

In hoofdstuk~\ref{ch:conclusie}, tenslotte, wordt de conclusie gegeven en een antwoord geformuleerd op de onderzoeksvragen. Daarbij wordt ook een aanzet gegeven voor toekomstig onderzoek binnen dit domein.