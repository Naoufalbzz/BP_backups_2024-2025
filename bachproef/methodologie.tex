%%=============================================================================
%% Methodologie
%%=============================================================================

\chapter{\IfLanguageName{dutch}{Methodologie}{Methodology}}%
\label{ch:methodologie}

%% TODO: In dit hoofstuk geef je een korte toelichting over hoe je te werk bent
%% gegaan. Verdeel je onderzoek in grote fasen, en licht in elke fase toe wat
%% de doelstelling was, welke deliverables daar uit gekomen zijn, en welke
%% onderzoeksmethoden je daarbij toegepast hebt. Verantwoord waarom je
%% op deze manier te werk gegaan bent.
%% 
%% Voorbeelden van zulke fasen zijn: literatuurstudie, opstellen van een
%% requirements-analyse, opstellen long-list (bij vergelijkende studie),
%% selectie van geschikte tools (bij vergelijkende studie, "short-list"),
%% opzetten testopstelling/PoC, uitvoeren testen en verzamelen
%% van resultaten, analyse van resultaten, ...
%%
%% !!!!! LET OP !!!!!
%%
%% Het is uitdrukkelijk NIET de bedoeling dat je het grootste deel van de corpus
%% van je bachelorproef in dit hoofstuk verwerkt! Dit hoofdstuk is eerder een
%% kort overzicht van je plan van aanpak.
%%
%% Maak voor elke fase (behalve het literatuuronderzoek) een NIEUW HOOFDSTUK aan
%% en geef het een gepaste titel.

Het onderzoek begint met een uitgebreide literatuurstudie over back-upstrategieën, ransomware-resistente opslag, en immutable storage. Hierbij wordt een overzicht gegeven van de state of the art, waarbij de nieuwste technieken en strategieën voor databeveiliging in kaart worden gebracht. Deze literatuurstudie biedt de fundamentele kennis die nodig is om het bestaande back-upplan te analyseren en geeft een goed beeld van hoe organisaties effectief hun back-upsystemen kunnen beveiligen. In de tweede fase zal de huidige back-upstrategie van Forvis Mazars worden geanalyseerd en verbeterd. Momenteel wordt er elke dag één volledige back-up door Azure automatisch uitgevoerd, wat zorgt voor een basisbeveiliging. Naast de automatische back-ups beschikt Forvis Mazars ook over een script dat manuele back-ups uitvoert, maar deze worden niet frequent genoeg gedaan en missen een geautomatiseerde structuur. Door de bestaande methode te optimaliseren, wordt zowel de veiligheid als de efficiëntie van het back-upproces verhoogd.
\subsection{Requirements-analyse}
\subsubsection{1. Must Have (Essentiële vereisten)}
Deze vereisten zijn cruciaal voor de verbetering van de back-upstrategie en moeten absoluut worden geïmplementeerd om een werkbare en veilige oplossing te garanderen:

\begin{itemize}
    \item \textbf{Automatisering van de manuele back-ups:} De huidige manuele back-ups moeten worden geautomatiseerd om de frequentie van back-ups te verhogen en de afhankelijkheid van menselijke interventie te verminderen. De automatische back-up moet dagelijks worden uitgevoerd en volledig geïntegreerd zijn in de bestaande Azure-omgeving van Forvis Mazars.
    
    \item \textbf{Beveiliging tegen ransomware:} De nieuwe back-upstrategie moet ransomware-resistent zijn, wat betekent dat back-ups moeten worden beschermd tegen externe aanvallen die de back-ups zelf kunnen infecteren. Dit vereist de implementatie van technieken zoals \textit{immutable storage}, zodat back-ups niet gewijzigd of verwijderd kunnen worden tijdens een ransomware-aanval.
    
    \item \textbf{Regelmatige en betrouwbare volledige back-ups:} Er moet gezorgd worden voor een betrouwbare en regelmatige uitvoering van volledige back-ups van de databases om te garanderen dat bij een systeemfout of cyberaanval altijd een up-to-date herstelpunt beschikbaar is.
    
    \item \textbf{Herstelcapaciteit (Restore from backup):} Het herstelproces moet efficiënt en snel kunnen worden uitgevoerd vanuit de back-ups. De back-upstrategie moet testen hoe snel en betrouwbaar de systemen kunnen worden hersteld in geval van dataverlies.
\end{itemize}

\subsubsection{2. Should Have (Aanbevolen vereisten)}
Deze vereisten dragen bij aan de effectiviteit van de back-upstrategie, maar zijn niet strikt noodzakelijk voor de eerste versie van de oplossing:

\begin{itemize}
    \item \textbf{Differentiële en incrementele back-ups:} Hoewel volledige back-ups cruciaal zijn, moeten incrementele en/of differentiële back-ups overwogen worden om de belasting op de opslagcapaciteit en netwerkinfrastructuur te verminderen. Dit kan bijdragen aan de optimalisatie van de back-upstrategie door slechts gewijzigde gegevens te back-uppen in plaats van de volledige dataset.
    
    \item \textbf{Documentatie:} Er moet gedetailleerde documentatie beschikbaar zijn over het back-upproces, de gebruikte technieken, en de herstelprocedures.
    
\end{itemize}

\subsubsection{3. Could Have (Wenselijke vereisten)}
Deze vereisten kunnen de back-upstrategie verder verbeteren, maar kunnen in eerste instantie worden uitgesteld als er beperkingen zijn in tijd of middelen:

\begin{itemize}
   
    \item \textbf{Versiebeheer van back-ups:} Het invoeren van versiebeheer voor back-ups maakt het mogelijk om verschillende versies van data op te slaan. Dit kan nuttig zijn voor het herstellen van data naar een specifieke eerdere versie (bijvoorbeeld na een fout die niet meteen werd opgemerkt).
    
    \item \textbf{Geautomatiseerd herstelproces:} Een geautomatiseerd herstelproces kan worden ontwikkeld, zodat de systemen automatisch kunnen worden hersteld in geval van dataverlies, wat de downtime minimaliseert.
\end{itemize}

\subsubsection{4. Won't Have (Niet noodzakelijke vereisten)}
Deze vereisten worden niet opgenomen in de huidige verbeteringsronde van de back-upstrategie vanwege beperkingen in tijd, middelen, of prioriteit:

\begin{itemize}
    \item \textbf{Complexe multi-cloud back-upoplossingen:} Hoewel multi-cloud back-upstrategieën voordelen kunnen bieden, is het implementeren van een complexe multi-cloud-oplossing voor Forvis Mazars op dit moment niet noodzakelijk, aangezien Azure al gebruikt wordt voor back-ups en de primaire focus ligt op het verbeteren van de huidige strategie binnen de Azure-omgeving.
    
    \item \textbf{Fysieke back-ups op externe schijven:} Aangezien Forvis Mazars gebruik maakt van een cloud-gebaseerde infrastructuur voor de back-ups, is het niet noodzakelijk om fysieke externe schijven of on-premise hardware-oplossingen in te zetten voor back-updoeleinden. Dit zou alleen meer complexiteit en kosten met zich meebrengen zonder aanzienlijke voordelen.
\end{itemize}

\subsubsection{Conclusie}
Deze requirementsanalyse geeft de noodzakelijke vereisten voor de verbetering van de back-upstrategie van Forvis Mazars weer, met een focus op beveiliging tegen ransomware, automatisering van de back-ups, en het herstelproces. Door de integratie van automatisering, immutable storage en verbeterde back-uptechnieken zal Forvis Mazars in staat zijn om zowel de veiligheid als de efficiëntie van hun gegevensbeheer te verhogen. Verdere verbeteringen, zoals incrementele back-ups en cloud-integratie, kunnen op een later moment worden geïmplementeerd, afhankelijk van de beschikbare middelen en de prioriteiten van het bedrijf.

\subsection{Proof-Of-Concept}
In de Proof-of-Concept zal er een virtuele omgeving opgezet worden in VirtualBox met immutable opslag. Binnen deze omgeving wordt een extra harde schijf geconfigureerd die als back-upschijf dient, met als doel aan te tonen hoe het implementeren van immutable storage kan helpen tegen ransomware-aanvallen. De PoC begint met het aanmaken van een nieuwe virtuele harde schijf in VirtualBox die enkel voor het opslaan van back-updata gebruikt zal worden. Deze schijf zal worden toegevoegd als tweede schijf aan de virtuele machine, zodat er een gescheiden opslagruimte voor back-ups beschikbaar is. Nadien wordt deze schijf van de virtuele machine ingesteld als een “read-only” schijf, zodat wijzigingen beperkt worden en data effectief beschermd is tegen ongewenste aanpassingen of verwijdering, dit simuleert immutable storage. Na het configureren van de read-only schijf, zal de back-updata opgeslagen worden op deze schijf. De back-updata blijft toegankelijk voor het systeem maar kan niet worden aangepast zonder speciale rechten, wat een basisniveau van immutabiliteit simuleert. In de ransomware-simulatiefase zal een testaanval worden uitgevoerd waarin bestanden op de primaire schijf worden versleuteld of verwijderd om het effect van een ransomware-aanval na te bootsen. Omdat de back-upschijf read-only is, zullen de bestanden op deze schijf intact blijven en ongewijzigd, wat de waarde van immutable storage aantoont voor herstel na een ransomware-aanval. Deze methode biedt een praktische Proof-of-Concept waarmee kan worden aangetoond hoe immutable storage kan bijdragen aan de beveiliging en integriteit van back-ups in een organisatieomgeving.